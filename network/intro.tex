\section{はじめに}

\subsection{ネットワークとは}

	「ネットワーク」とは,なにかとなにかが網の目のようになにかによって繋がっている状態のことであり,なにかを運ぶためのものである.
	単にネットワークというと,物流・道路・神経などいろいろなものがあるが,特に「コンピュータネットワーク」においては,コンピュータとコンピュータが網の目のようにケーブルなどの通信媒体によって繋がっており,データを運ぶためのものであると定義できる.

\subsection{ネットワークの利点}

	コンピュータネットワークを用いると,メールやファイルのやり取りが行えるだけではなく,他のコンピュータに繋がっているプリンタを使用したり,他のコンピュータにデータを処理させることもできる.

	メールやファイル,印刷したいデータなど,コンピュータやユーザが持つものを「リソース」と呼ぶ.
コンピュータネットワークを利用する最大の利点は,こうしたリソースを複数のコンピュータで共有することであるといえる.


\subsection{プロトコル}
	人と人とが会話をするとき,お互いが異なる言語を用いてしまうと正しく意思疎通を行うことができない.つまり,人間同士の会話が成立するためには「お互いが理解できる言語を用いる」という暗黙的なルールが存在していなければならない.
	
	同様に,コンピュータの世界においても,複数のコンピュータ同士がネットワークを介して通信を行うときには共通のルールが必要となる.柔軟な対応ができる人間とは異なり,コンピュータは機械であるため,通信を行うためには事前にありとあらゆるルールを決めておく必要がある.
	
	例えば,メールを送信するときのルールやファイルのやり取りを行うときのルール,通信相手を特定するときのルール,通信経路を選択するときのルール,通信途中でデータが壊れてしまったときのルール,更には,通信を行う際に用いるケーブルの種類や電気信号の変換方法など,様々な取り決めやルールが存在している.これらのルールのことを「プロトコル」と呼び,使用するプロトコルが異なれば正しく通信を行うことができない.