\section{OSI参照モデル}

\subsection{OSI参照モデル}
	ネットワーク機器や端末を開発するメーカーがそれぞれ好き勝手なプロトコルを用いてしまうと,メーカーの異なる機器同士で通信が行えなくなってしまうという問題が起こる.そこで,世界的にプロトコルの標準化が進められ,「OSI参照モデル」というプロトコルの設計モデルが1984年に国際標準化機構(ISO)によって制定された.
	
	現在では,プロトコルの設計モデルとして「TCP/IPモデル」が一般的に普及している.TCP/IPモデルは必ずしもOSI参照モデルに準拠している訳ではないが,データ通信における流れとしてはOSI参照モデルと類似している部分が多い.そのため,OSI参照モデルはネットワーク通信を行う際の基本的な考え方であるとして今でも広く浸透している.今回の講習でも,分かりやすさのためにOSI参照モデルを用いて説明を行う.


\subsection{階層構造}
	一度のデータ通信を行うとき,必要となるプロトコルは一つだけではなく,複数のプロトコルが使用されている.
	OSI参照モデルでは,それぞれのプロトコルの役割に応じてデータ通信を七つの段階に分類している.各段階のことを「レイヤ」と呼び,ネットワークによるデータ通信は各レイヤごとの複数のプロトコルで実現される.OSI参照モデルにおける各レイヤの名称と役割を表\ref{tab:osi}に示す.

	\begin{table}[htb]
		\begin{center}
			\caption{OSI階層モデルの各レイヤと役割}
			\begin{tabular}{|c|c|c|} \hline
				レイヤ & 名称 & 役割 \\ \hline
				レイヤ7 & アプリケーション層 & アプリケーションソフト間での通信を規定 \\ 
				レイヤ6 & プレゼンテーション層 & データ形式に関する規定 \\
				レイヤ5 & セッション層 & 通信の開始/終了に関する規定 \\
				レイヤ4 & トランスポート層 & 通信の品質を確保するための通信手順を規定 \\
				レイヤ3 & ネットワーク層 & 異なるネットワーク間の通信を規定 \\
				レイヤ2 & データリンク層 & 同じネットワーク内の通信を規定 \\
				レイヤ1 & 物理層 & 接続のための物理的な規定 \\ \hline
			\end{tabular}
			\label{tab:osi}
		\end{center}
	\end{table}

	各レイヤは独立しており,レイヤのプロトコルの変更は他のレイヤに影響を及ぼさない.また,基本的には下位のレイヤは上位のレイヤを考慮して設計されており,上位のレイヤは下位のレイヤを意識する必要は無い.


\subsection{カプセル化}

	誰かに宅配便を送る場合,送る側がまず贈り物を梱包材に包み,ダンボールに入れて,宛先や送り主を書いた配達表を貼ることで,配達員に宛先まで運んでもらう.宅配便を受け取った側は,配達表を剥がし,ダンボールから取り出して,梱包を解くという,送る側と逆の手順を踏むことで贈り物を正しく受け取ることができる.
	
	同様に,コンピュータがデータ通信を行う場合も,送りたいデータに対して制御情報(ヘッダ)を付け加える作業が必要となる.制御情報とは,送信先・送信元のアドレスや,それぞれのプロトコルで必要となる情報などである.
	
	OSI参照モデルでは,送信側が各レイヤにおいて7から1の順番でそれぞれの手順をこなし,送りたいデータに制御情報を付け加えていく.このように,各レイヤにおいてデータに制御情報を付け加えていく作業を「カプセル化」と呼ぶ.受信側は,各レイヤにおいて1から7の順番でそれぞれの手順をこなし,送信側とは逆の順序で制御情報を取り除いていくことで,最終的に目的のデータを受け取ることができる.
	OSI参照モデルにおけるカプセル化を用いたデータ通信の流れを図\ref{fig:capsule}に示す.

	\begin{figure}[htb]
		\begin{center}
			\includegraphics[width=160mm]{osi_capsule.eps}
		\end{center}
		\caption{OSI参照モデルによるカプセル化の流れ}
		 \label{fig:capsule}
	\end{figure}


\subsection{TCP/IPモデル}
		
	現在インターネットで広く用いられているプロトコル群はIETFが制定したTCP/IPプロトコル群であり,これらは「TCP/IPモデル」がベースとなっている.OSI参照モデルが七つのレイヤで構成されていたのに対し,TCP/IPモデルは四つのレイヤで構成される.OSI参照モデルとTCP/IPモデルの階層の対応を図\ref{fig:tcp_layer}に,また,TCP/IPで用いられるプロトコルの例を表\ref{tab:tcp}に示す.
	
	\begin{figure}[htb]
		\begin{center}
			\includegraphics[width=160mm]{tcp_layer.eps}
		\end{center}
		\caption{OSI参照モデルの7階層とTCP/IPモデルの4階層}
		 \label{fig:tcp_layer}
	\end{figure}

	\begin{table}[htb]
		\begin{center}
			\caption{TCP/IPモデルのプロトコル例}
			\begin{tabular}{|c|c|c|} \hline
				レイヤ & 名称 & プロトコルの例 \\ \hline
				レイヤ4 & アプリケーション層 & HTTP,FTP,SMTP \\
				レイヤ3 & トランスポート層 & TCP,UDP \\
				レイヤ2 & インターネット層 & IP,ARP \\
				レイヤ1 & インターフェース層 & Ethernet,PPP \\ \hline
			\end{tabular}
			\label{tab:tcp}
		\end{center}
	\end{table}
	
	TCP/IPモデルのレイヤ4は,OSI参照モデルでいうところのレイヤ5,6,7の部分に対応し,また,TCP/IPモデルのレイヤ1は,OSI参照モデルでいうところのレイヤ1,2の部分に対応している.注意してほしいのは,実際にはTCP/IPモデルとOSI参照モデルはまったく別のプロトコル設計モデルだということである.例えば.TCP/IPモデルのインターネット層とOSI参照モデルのネットワーク層は,役割は似ているが同一のものではない.
	
	以降では,基本的にTCP/IPプロトコル群を利用したネットワークについて説明する.
	
	
	