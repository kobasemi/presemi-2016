\documentclass[11pt]{jarticle}
\usepackage[dvipdfmx]{graphicx}        % 画像を扱えるようにする.
\usepackage{ascmac}                    % 丸枠を使用可能にする.
\usepackage{color}                     % 色文字を有効にする.
\usepackage{url}                       % \url{}を使用可能にする.
\usepackage{fancyvrb}                  % \Verbatimを使用可能にする.
\usepackage{fancybox}                  % itembox等を使用可能にする.
\usepackage[junior]{gaiyo}             % styファイルは既存のものを使用.
\usepackage{eclbkbox}
%
% ここからマニュアル用スタイル設定.
\makeatletter
\gdef\@id{}
\gdef\@author{}
%
\def\hdtype#1{\gdef\hd@type{#1}}
\def\@maketitle{\begin{center}
  \vspace{\baselineskip}
  {\fontsize{16pt}{0.6565cm}\selectfont \bfseries
    \@title \par
  }
\end{center}
}
\makeatother
% ここまでマニュアル用スタイル設定.
%
% タイトルとヘッダの設定.
\title{第1回 プレゼミ資料}
\hdtype{第1回 プレゼミ資料}
%
% ここから文書開始.
%
\begin{document}
\maketitle
%
% ここから本文.
% 
\section*{はじめに}
ゼミ活動は,大学生活の半分の期間を占める重要なものである.大学で何を学んだかということは,ゼミで何をしたかとほぼ同義であると言える.それゆえ,今後の学生生活はゼミ活動を中心に回っていくといっても過言ではない.ここでは,小林ゼミにおいてゼミ活動を行うにあたり,必要となる知識等を学んでもらう.

\section*{午前の部}
\section{小林ゼミについて}
\subsection{小林ゼミの掟}
大学には様々な環境で育ち,様々な考え方を持った人が集まるため,ゼミに対する考え方などの食い違いが発生し,活動が円滑に行われない可能性もある.それを避けるためにも,小林ゼミでは予めいくつかのルールが定められており,明文化されている.以下に特に重要なものを挙げておく.かなり厳しいことと感じることかもしれないが,これらのルールを厳守することは決して今後の人生においてマイナスとなることは無いため,厳守するようにして欲しい.

\begin{boxnote}
	\begin{itemize}
		\item 「ほう・れん・そう」(報告・連絡・相談)はきちんとしよう.
		\item ゼミ生同士で議論することを避けるな.
		\item テーマは一人一つずつ.しかし,一人だけで進めるな.
		\item \underline{3回生,4回生,大学院生を独立して運用するつもりはない}ので,お互いに協同作業をこころがけること.
		\item 卒業研究・専門演習で研究をすすめるにあたって,週1回のゼミの時間は指導時間としては十分ではないので,小林・大学院生の空き時間を狙ってミーティングすること(Slack含む)
		\item \underline{「ゼミの時間以外の活動が最も重要である」}ことを認識すること.
		\item 勉強会への参加など,自分に足りない能力・技術を磨く努力をすること.
		\item 誰かに相談するときには,口頭ではなくペーパーを書いて持って行くこと.
		\item 本をたくさん読んで,文章力を身につけること.
		\item 最後に,小林は,ゼミ担当者として,ゼミ生を指導して卒業させる義務はある.しかし,\underline{ゼミに来ない学生に対して指導しなければならない義理はない.}
	\end{itemize}
\end{boxnote}

\vspace{0.5cm}

\subsection{ゼミ活動}

小林ゼミでの活動は,似通ったテーマの研究をしている人同士で集まった「研究グループ」ごとに進める予定である.
プレゼミ最終日に,現在小林ゼミで行われている研究を紹介する「研究紹介」を見て貰った上で,どの研究グループに所属するかを決めてもらう.
そして,今後は主にその研究グループ内で連携しながら,グループのテーマに関する知識や理解を深め,研究を進めていく.

\subsection{卒業研究}
3年次のゼミの時間は前述のとおり専門演習となっているが,4年次になるとこれが卒業研究となる.卒業\underline{研究}なので,自分で何らかのテーマを決め,実験を行い,その結果と考察などを論文としてまとめて提出する必要がある.テーマはゼミの内容に沿い,かつ新しい技術や手法を用いているものならば何でも良い.また,過去の小林ゼミ卒業生が行った研究を引き継ぎ,既存の問題点を解消する手法を提案し,実際に実装を試みたり,別の方向からアプローチを行うというのも可能である.卒業研究に関して共通して言えることは,\underline{しっかりと結果を出さなければならない}ということである.その為には様々な文献に目を通し,新たな知識を身につけ,何度も繰り返し実験を行う必要がある.もちろん全てがトントン拍子に上手くいくことは無いので,研究に行き詰ったときは,ゼミ室に来て誰かに相談したり,大学院生や先生とミーティングを行うことが不可欠である.

\subsection{卒業研究に関する今後の予定}
研究結果は,4年の冬の卒論提出時期に全てまとめて出せば良いというわけではない.小林ゼミでは毎週のゼミ報告会に加えて定期的に全体報告会を行っており,進捗状況や実験結果を報告している.注意しなければならないのは,\underline{研究はこれらの報告会のために行うものではない},ということである.報告会が近くなってから慌てて研究を行っても,完成度が高いものが作れる可能性は0である.つまり,研究はこれらの日程を考慮した上で,計画的に進めていかなければならない.ルールブックにも記載されているが,以下に小林ゼミ学部生の発表会などの主なイベントと.その内容について簡単に説明する.

\vspace{0.5cm}

\begin{breakbox}
	\begin{itemize}
	\item {\bf 9月中旬:夏の全体報告会(全員)}
		\begin{itemize}
		\item 1人15分(10分発表,5分質疑).質問は必ず行うこと(ノルマ:1人4回以上).
		\item 発表用にプレゼンテーションを用意しておくこと.
		\item また,原稿として研究概要をA4用紙2ページ(1ページ+数行等は却下)にまとめて,事前に添削を受け,提出すること.
		\item 原稿は\LaTeX で,指定のスタイルファイル(卒業研究概要用)を用いて書くこと.
		\end{itemize}
	\item {\bf 12月中旬:卒業研究中間発表会(学部4年)}
		\begin{itemize}
		\item 発表時間は時間無制限一本勝負.
		\item \underline{「卒論タイトルが決まっていない」,「結果が出てない」は却下(={\bf 卒業できない})}.
		\item この段階で卒論のテーマを変えない(変えるならもう1年が必要).
		\end{itemize}
	\item {\bf 1月上旬:卒業研究概要提出(学部4年)}
		\begin{itemize}
		\item 当然ながら,「卒業研究の概要」であるので,「概要」を書くことに専念しないこと.
		\item 中間発表までに執筆した卒業論文の草稿+その後の進捗をまとめて圧縮すること.
		\item 卒業研究概要の提出後,卒業研究発表会までに卒業論文を仕上げる.1度や2度の添削で完成するようなものではない.
		\end{itemize}
	\item {\bf 1月上旬:プレ卒論提出(学部3年)}
		\begin{itemize}
		\item プレ卒論は,そのときにやっているテーマで執筆すること(最大8ページ).
		\item 1度提出すれば終わりではなく,先生と大学院生の添削を受け,修正・加筆をOKが出るまで繰り返す.
		\item この時添削結果等の過去の履歴を消さないこと.
		\end{itemize}
	\item {\bf 2月中旬:卒業研究発表会(学部4年)}
		\begin{itemize}
		\item 1人15分(10分発表,5分質疑).
		\item 予め他のゼミ生の前で発表の練習をしておくこと.
		\item 発表会は何が何でも参加すること(就職関連・病気・怪我も含めてあらゆる事情を考慮しない).
		\item 発表会後,卒論で使用したシステム・プログラム・データ等,卒業研究概要及び卒業論文の原稿を全て,ファイルサーバに集約すること.
		\item 卒業論文及びデータ類は,「小林ゼミ」に対して提出すると思うこと.
		\end{itemize}
	\item {\bf 3月下旬:研究紹介(全員)}
		\begin{itemize}
		\item 4月に入ってくる新3回生の前でプレ卒論の内容を発表.
		\item 夏の全体報告会と同様に,プレゼンテーションと原稿を用意すること.
		\end{itemize}
	\end{itemize}
\end{breakbox}


\section{ゼミシステム}
\subsection{ネットワーク構成}
小林ゼミでは,研究を快適に進めるために独自のネットワークを構築している.ゼミネットワークは,大学院棟3階プロジェクトスペース(以下,大学院棟),情報演習棟(K棟)2階ゼミ室(以下,ゼミ室),そしてA棟にある小林先生の個人研究室から成る.各内部ネットワークは相互に利用できるようになっており,\verb|netmask|は16ビットにして比較的大きなネットワークとしている.また,3箇所全てには無線LANのアクセスポイントが配置されており,WPA2+PSKで暗号化が施されている.SSID,事前共有鍵は同一のものにしてあり,どこか1箇所で設定すれば,残りの2箇所でも利用することができる.


%昔はそれぞれの場所で独立したネットワークを構築していたが,利便性の悪さが問題になっていたので,PPTPによるVPN機能の付いたルーターによって相互に利用できるようにした.しかし,VPNの性能があまり良くなかったため,ルーター自体を置き換えて,Ethernet over IP によって直接利用ができるようにした.このようにして,現在のネットワーク構成ができあがった.


いずれのネットワークにも,インターネット(自宅など)から接続できるようになっているが,一度大学院棟にある\verb|cririn|に接続するか,ゼミネットへVPN(Virtual Private Network)接続する必要がある.

%\subsubsection{VPN}
%VPNとは,Virtual Private Networkの略であり,異なるネットワークをあたかも同じ場所にあるかのように見せるネットワーク技術(仮想的なプライベートネットワーク接続)のことである.論理的なつながりであり,物理的なつながりは無い.VPNは,L2TP,PPTP,MPLSなど,様々な方式が考案・実用化されている.VPNは暗号化とトンネリングの2つの技術から成り立っている.VPN接続にはLAN間接続VPNとリモートアクセスVPNの2つの利用形態がある.LAN間接続VPNは例えば,A地点とB地点のLANを接続するものであり,小林ゼミのネットワークはこの形態にあたる.

%\subsubsection{トンネリング}
%トンネリングとは,特定の目的のための通信路を確立する技術の総称である.トンネリングを使うと,全ての通信をルータが集約して暗号化できるので,端末が個別に暗号化する必要が無くなる.

\subsection{ネットワークアドレス}
ゼミの拠点は3つあり,それぞれにネットワークアドレスが割り当てられている.さらに,各箇所でDHCPを有効にしており,別の拠点でそのアドレスを使わないようにするための制御も行われている.拠点ごとのネットワークアドレスを表1に示す.表1はそれぞれの内側で使用するアドレスになっているが,当然ながらWAN側(外側)のアドレスも存在する.拠点ごとのWAN側のIPアドレスを表2に示す.

\begin{table}[htbp]
\begin{center}
\begin{tabular}{cc}
\begin{minipage}[t]{0.5\hsize}
\begin{center}
\caption{拠点ごとのネットワークアドレス}
\begin{tabular}{|c|c|}
  \hline
  拠点 & ネットワークアドレス \\ \hline \hline
  A棟 & 10.1.1.0/16 \\ \hline
  D棟 & 10.1.3.0/16 \\ \hline
  K棟 & 10.1.4.0/16 \\ \hline
  D棟DHCP & 10.1.5.0/16 \\ \hline
  K棟DHCP & 10.1.6.0/16 \\ \hline
  個人割り当て & 10.1.100.0/16 以降\\ \hline
\end{tabular}
\end{center}
\end{minipage}
\begin{minipage}[t]{0.5\hsize}
\begin{center}
\caption{WAN側のIPアドレス}
\begin{tabular}{|c|c|}
  \hline
  拠点 & IPアドレス \\ \hline \hline
  A棟 & 158.217.42.100/24 \\ \hline
  D棟 & 158.217.77.225/24 \\ \hline
  K棟 & 158.217.62.103/24 \\ \hline
\end{tabular}
\end{center}
\end{minipage}
\end{tabular}
\end{center}
\end{table}

\newpage

\subsection{サーバ群}
大学院棟には\verb|cririn|というメインのサーバが存在し,\verb|cririn|ではSSH(Secure SHell)サービスが稼働している.また,ゼミのWebサーバとして\verb|melon|が存在し,ここではWebサービスとMySQLサービスが稼働している.\verb|melon|で稼動しているWebサーバでは,ゼミのホームページを開設しており,CMS(Contents Management System)としてXOOPSを利用している.他にはDNSサーバとして\verb|sakura|や\verb|tsubame|などが挙げられる.これらのサーバには,表\ref{servers}に示すアドレスで接続することができる.FQDN(Fully Qualified Domain Name:完全修飾ドメイン名)で示したものは外部から接続できるもの,IPアドレスで示したものはゼミネットワーク内からのみ接続できるものである.

\subsection{仮想マシン用サーバ}
ゼミの実験システムはほとんどが仮想マシンでできており,卒業研究も仮想マシン上で行っていることが多い.それらの仮想マシンを管理しているのが,VMM(Visual Machine Monitor)である.VMMとして使用しているのはVMware ESXiであり,目的のVMMにHTTPSで接続することにより,管理クライアントソフトウェアをダウンロードすることができる.ゼミネットワークにはVMMがいくつか存在し,使用者及びその目的によって棲み分けがなされている.ゼミネットワークに存在するVMMを表\ref{vmm}に示す.今後の基本的な使い方としては,VMMを統括しているサーバが\texttt{10.1.3.30}に存在するのでここへアクセスし,その後10.1.3.36へアクセスするという形になる.これについても次回以降のプレゼミで触れる.

\begin{table}[htbp]
\begin{center}
\caption{主なサーバ群}
\label{servers}
\begin{tabular}{|c|c|c|}
	\hline
	サーバ名 & アドレス & 用途 \\ \hline \hline
	\verb|cririn| & \verb|cririn.firefly.kutc.kansai-u.ac.jp| & SSHサーバ \\ \hline
	\verb|melon| & \verb|www.firefly.kutc.kansai-u.ac.jp| & Webサーバ \\ \hline
  \verb|ookini| & \verb|10.1.3.40| & Acrive Directory,VPNサーバ \\ \hline
	\verb|sakura| & \verb|10.1.3.21| & 1st DNS,DHCP,LDAPサーバ\\ \hline
	\verb|tsubame| & \verb|10.1.3.80| & 2nd DNSサーバ\\ \hline
	\verb|zeon| & \verb|10.1.3.30| & ゼミ生用VMM統括サーバ(vCenter) \\ \hline
	\verb|belka| & \verb|10.1.3.31| & 先生\&ゼミ基幹システム用  \\
	  &  & VMM統括サーバ(vCenter) \\ \hline
\end{tabular}
\end{center}
\end{table}

\begin{table}[htbp]
\begin{center}
\caption{VMM}
\label{vmm}
\begin{tabular}{|c|c|c|}
	\hline
	サーバ名 & アドレス & 用途 \\ \hline \hline	
	\verb|yotta| & \verb|10.1.1.12| & 小林先生の研究用\\ \hline
	\verb|zetta| & \verb|10.1.1.13| & 小林先生の研究用\\ \hline
	\verb|lain| & \verb|10.1.3.33| &  学部生\&大学院生用 \\ \hline
	\verb|penguin| & \verb|10.1.3.36| & 学部生用 \\ \hline
	\verb|rakko| & \verb|10.1.3.38| & 大学院生用 \\ \hline
	\verb|miku| & \verb|10.1.3.39| & ゼミ基幹サービス用 \\ \hline
\end{tabular}
\end{center}
\end{table}

\section{仮想マシン希望IPの調査}

小林ゼミでは,ゼミ生1人1人が研究室のネットワークで使用できるローカルIPアドレスとして,1人1つのセグメントが割り当てられている.具体的には,小林ゼミのネットワークのIPアドレスは「10.1.x.y」となっており,「x」の部分を1人1つ選ぶことが出来る.そして「y」の部分は自分の研究用サーバに自由に割り振れるため,253台まで作成することができる.

IPアドレスの「x」の部分について,現在使用されていない番号の中から希望する数字を1つ選び,配布する紙に記入してほしい.この番号は今後使うことになるので,忘れないように.

\section{希望IDの調査}

ゼミHPの認証や,各種ゼミシステムを利用するときは,ゼミシステムで有効なユーザIDが必要となる.
希望がなければ学籍番号(kXXXXXX)となるが,別のユーザIDが良いという場合は,それを教えてほしい.

\section{個人用端末貸出し希望調査}

来週は実際にゼミシステムを使用するため,ノートPCを持参すること.
ノートPCを持っていない人,持っているが性能がよろしくない人については,ゼミで貸し出しができるので,Windowsが良いかMacが良いかを教えてほしい.


\section*{午後の部}

\section{LANケーブル作り}

LANケーブルは市販されている物を使っている人が大半であるが,実は材料と工具さえあれば簡単に自分で作ることができる.小林ゼミはネットワークに関する研究を行っており,LANケーブルを頻繁に使用する.そのため,小林ゼミで活動するならば,自分でLANケーブルを作れるようになっておくと便利である.そこで,本日はカテゴリー5eケーブルとRJ45プラグを使い,実際にLANケーブルの作成を行う.

\subsection{LANケーブルの作り方}
LANケーブルはケーブルとプラグを材料として,「かしめ工具」を使って作成する.かしめ工具はプラグをケーブルに圧着するための道具である.小林ゼミ所有のものにはカッターが付いているのでケーブルを切ることもできる.

LANケーブルの作成手順は以下の通りである.

\begin{enumerate}
 \item ケーブルを適当な長さに切り分ける.
 \item カッターを使ってケーブルの皮膜を3cm程度取り除く.
 \item 芯線のよりをほどき,直線になるようにまっすぐ伸ばす.
 \item 左から,{\it 白橙,橙,白緑,青,白青,緑,白茶,茶}となるように芯線を並べる.
 \item 芯線の長さを切り揃える.
 \item プラグの爪の無い面を上にして,芯線を奥まで差し込む.
 \item かしめ工具を使ってケーブルにプラグを圧着する.
 \item ケーブルの反対側も同じようにプラグを取り付ける.
\end{enumerate}

LANケーブルが完成したらケーブルテスターでチェックする.問題がなければ,すぐにでも使用することができる.

\newpage

\section{今後の予定}

\begin{description}
 \item[3月10日 10:00〜]\mbox{}\\ 
 	ネットワークの基礎
 \item[3月17日 10:00〜]\mbox{}\\
	TeX,仮想マシン構築
 \item[3月24日 10:00〜]\mbox{}\\
	Shell,プログラミング
 \item[3月31日 13:00〜]\mbox{}\\
 	研究紹介
\end{description}

※2日目のネットワーク講習でWiresharkというソフトウェアを使います.自前のPCを持ってくる人は事前にインストールしてくるように.

※3日目と4日目の内容が入れ替わったので注意.

\begin{itembox}[l]{Memo}
 \\
 \\
 \\
 \\
 \\
 \\
 \\
 \\
 \\
 \\
 \\
 \\
 \\
 \\
 \\
 \\
 \\
 \\
 \\
 \\
 \\
\end{itembox}



%
%
% 編集者記録欄(学籍番号及び氏名).
\vspace{1cm}
\begin{table}[b]
  \begin{flushright}
    \begin{tabular}{llll}
      2015年3月5日 & 初版 & 14M7122 & 吉井 章 \\
      2016年3月3日 & 第二版 & 12-367 & 坂東 翼
    \end{tabular}
  \end{flushright}
\end{table}

\end{document}
